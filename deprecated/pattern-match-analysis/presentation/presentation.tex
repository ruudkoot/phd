\documentclass{beamer}

%% Packages %%%%%%%%%%%%%%%%%%%%%%%%%%%%%%%%%%%%%%%%%%%%%%%%%%%%%%%%%%%%%%%%%%%%

\usepackage{amsfonts}
\usepackage{amsmath}
\usepackage{amssymb}
\usepackage{mathtools}
\usepackage{xspace}

\usepackage{graphicx}
\usepackage{tikz}

\usepackage{alltt}                      %% Replace by lhs2TeX...


%% Theme %%%%%%%%%%%%%%%%%%%%%%%%%%%%%%%%%%%%%%%%%%%%%%%%%%%%%%%%%%%%%%%%%%%%%%%

\usetheme{Darmstadt}

%% lhs2TeX %%%%%%%%%%%%%%%%%%%%%%%%%%%%%%%%%%%%%%%%%%%%%%%%%%%%%%%%%%%%%%%%%%%%%

%include polycode.fmt

%format undefined = "\ensuremath{undefined}"

%format leadsto = "\ensuremath{\leadsto}"
%format ldots = "\ensuremath{\xspace\ldots\xspace}"
%format x_1 = "\ensuremath{x_1}"
%format x_2 = "\ensuremath{x_2}"
%format x_i = "\ensuremath{x_i}"
%format x_j = "\ensuremath{x_j}"
%format x_1 = "\ensuremath{x_1}"
%format xr  = "\ensuremath{x_\mathrm{r}}"
%format pi_i = "\ensuremath{\mathbf{\pi}_i}"
%format pi_j = "\ensuremath{\pi_j}"
%format letrec = "\ensuremath{\mathbf{letrec}}"

\newcommand{\tauhat}{\widehat{\tau}}

\newcommand{\fresh}{\mbox{\ fresh}}
\newcommand{\also}{\ \&\ }
\newcommand{\TypeRel}[5][\widehat\Gamma]{#1 \vdash #2 : #3 \leadsto #4 \also #5}

\newcommand{\Bool}{\textbf{Bool}}
\newcommand{\TTrue}{\textbf{True}}
\newcommand{\FFalse}{\textbf{False}}
\newcommand{\TTr}{\textbf{T}}
\newcommand{\FFa}{\textbf{F}}
\newcommand{\Int}{\textbf{Int}}
\newcommand{\TupleOf}[2]{\textbf{(}#1, #2\textbf{)}}
\newcommand{\ListOf}[1]{\textbf{[}#1\textbf{]}}

\newcommand{\Top}{\top}

\newcommand{\Nill}{\textbf{[]}}
\newcommand{\Conss}[1]{\textbf{(\,\_\,:}#1\textbf{)}}

\newcommand{\id}{\mathrm{id}}
\newcommand{\leet}{\mathbf{let}\ }
\newcommand{\iin}{\mathbf{in}\ }

\newcommand{\ftv}{\mathrm{ftv}}
\newcommand{\unify}{\widehat{\mathcal{U}}}
\newcommand{\uniphi}{\mathcal{V}}

%%%%%%%%%%%%%%%%%%%%%%%%%%%%%%%%%%%%%%%%%%%%%%%%%%%%%%%%%%%%%%%%%%%%%%%%%%%%%%%%
%% Document %%%%%%%%%%%%%%%%%%%%%%%%%%%%%%%%%%%%%%%%%%%%%%%%%%%%%%%%%%%%%%%%%%%%
%%%%%%%%%%%%%%%%%%%%%%%%%%%%%%%%%%%%%%%%%%%%%%%%%%%%%%%%%%%%%%%%%%%%%%%%%%%%%%%%

\begin{document}

\title{Pattern Match Analysis}
\subtitle{For Higher-Order Languages}
\author{Ruud Koot}
\date {28 August 2012}

\maketitle

\section{Introduction} %%%%%%%%%%%%%%%%%%%%%%%%%%%%%%%%%%%%%%%%%%%%%%%%%%%%%%%%%

\subsection{Right?} %%%%%%%%%%%%%%%%%%%%%%%%%%%%%%%%%%%%%%%%%%%%%%%%%%%%%%%%

\begin{frame}%{What's new? (Compared to ordinary ADTs)}
\begin{quote}
Well-typed programs cannot ``go wrong''.

\par\noindent\hfill ---Robin Milner, 1978
\end{quote}
\end{frame}

\subsection{Wrong!} %%%%%%%%%%%%%%%%%%%%%%%%%%%%%%%%%%%%%%%%%%%%%%%%%%%%%%%%

\begin{frame}[fragile]
\begin{verbatim}
*** Exception: Non-exhaustive patterns in function f
\end{verbatim}
\end{frame}

\subsection{Examples} %%%%%%%%%%%%%%%%%%%%%%%%%%%%%%%%%%%%%%%%%%%%%%%%%%%%%%%%%%

\begin{frame}[fragile]{Partial Functions}
\begin{code}
head (x : xs) = x
\end{code}
\pause
\begin{code}
main =  let xs = if length "foo" > 5 then [1,2,3] else []
        in head xs
\end{code}
\pause
\begin{verbatim}
On line 2 you applied the function "head" to the empty
list "xs". The function "head" expects a non-empty list
as its first argument.
\end{verbatim}
\end{frame}

\begin{frame}{Compiler Construction}
\begin{code}
desugar :: ComplexAST -> SimpleAST
\end{code}
\pause
\begin{code}
desugar :: AST -> AST
\end{code}
\end{frame}

\begin{frame}{Invariants (1)}
\begin{code}
type Bitstring = [Int]

add :: Bitstring -> Bitstring -> Bitstring
add []      y      = y
add x       []     = x
add (0:x)   (0:y)  = 0 : add x y
add (0:x)   (1:y)  = 1 : add x y
add (1:x)   (0:y)  = 1 : add x y
add (1:x)   (1:y)  = 0 : add (add [1] x) y
\end{code}
\end{frame}

\begin{frame}{Invariants (2)}
\begin{code}
risers :: Ord a => [a] -> [[a]]
risers []            =  []
risers [x]           =  [[x]]
risers (x_1:x_2:xs)  =  let (s:ss) = risers (x_2:xs)
                        in if x_1 <= x_2  then  (x_1:s):ss
                                          else  [x_1]:(s:ss)
\end{code}

\begin{block}{Computes monotonically increasing segments}
\centering
\begin{code}
risers [1,3,5,1,2] leadsto [[1,3,5],[1,2]]
\end{code}
\end{block}
\end{frame}

\subsection{Related Work} %%%%%%%%%%%%%%%%%%%%%%%%%%%%%%%%%%%%%%%%%%%%%%%%%%%%%%

\begin{frame}{Related Work}
\begin{description}
\item[Dependent ML (Xi)] \hfill\\ Dependent-types over a decidable domain
\pause
\item[Static Contract Checking (Xu)] \hfill\\ Contracts specified in Haskell
\pause
\item[Catch (Mitchell)] \hfill\\Constraint-based, first-order language only
\end{description}
\end{frame}

\section{Overview} %%%%%%%%%%%%%%%%%%%%%%%%%%%%%%%%%%%%%%%%%%%%%%%%%%%%%%%%%%%%%

\subsection{Refinements} %%%%%%%%%%%%%%%%%%%%%%%%%%%%%%%%%%%%%%%%%%%%%%%%%%%%%%%

\begin{frame}{Examples}
\begin{eqnarray*}
\TTrue &:& \Bool^{\{\TTrue\}} \\ \pause
42 &:& \Int^{\{42\}} \\ \pause
\TupleOf{7}{\FFalse} &:& \TupleOf{\Int^{\{7\}}}{\Bool^{\{\FFalse\}}}^\top \\\pause
\textbf{[}3,2,1\textbf{]} &:& \textbf{[}\Int^{\{1,2,3\}}\textbf{]}^{\{\textbf{(\_\,:\,\_\,:\,\_\,:\,[])}\}} \\ \pause
\lambda x.\ x + 1 &:& \Int^\top \overset{\top}{\to} \Int^\top
\end{eqnarray*}
\end{frame}

\begin{frame}{Higher-Order Functions}
\begin{block}{Program}
\begin{code}
main b f =  if b then
                if  f 42  then  100  else  200
            else
                if  f 43  then  300  else  400
\end{code}
\end{block}
\pause
\begin{block}{Type}
\begin{overlayarea}{\textwidth}{.2\textheight}
    \only<2>{\[ \Bool \to (\Int \to \Bool) \to \Int \]}
    \only<3>{\[ \Bool^{\{\TTr, \FFa\}} \to (\Int^{\{42, 43\}}\to\Bool^{\{\TTr,\FFa\}})_{-}\to \Int^{\{100,200,300,400\}}\]}
    \only<4>{\[ \Bool^{\{\TTr, \FFa\}}_{-} \to (\Int^{\{42, 43\}}_{+}\to\Bool^{\{\TTr,\FFa\}}_{-})_{-}\to \Int^{\{100,200,300,400\}}_{+}\]}
    \only<5>{\[ \Bool^{\{\TTr\}}_{-} \to (\Int^{\{41, 42, 43\}}_{+}\to\Bool^{\{\FFa\}}_{-})_{-}\to \Int^{\{100,200,300,400,500\}}_{+}\]}
\end{overlayarea}
\end{block}
\end{frame}

\subsection{Abstract Values}

\begin{frame}{Integers}
\[ \mathsf{Sign} ::= +\ ||\ 0\ ||\ - \]
\pause
\[ \mathsf{Parity} ::= \mathsf{Even}\ ||\ \mathsf{Odd} \]

\[ \ldots \]
\end{frame}

\begin{frame}{Lists}
\[ \mathsf{Shape} ::= \Nill\ ||\ \Conss{\mathsf{Shape}}\ ||\ \star \]
\pause
\[ \top_\mathsf{Shape} = \{ \Nill, \Conss{\Nill}, \Conss{\Conss{\Nill}}, \Conss{\Conss{\star}} \} \]
\end{frame}

\section{Analysis} %%%%%%%%%%%%%%%%%%%%%%%%%%%%%%%%%%%%%%%%%%%%%%%%%%%%%%%%%%%%%

\subsection{Overview}

\begin{frame}{Overview}
\begin{enumerate}
\item Generate constraints
\item Solve constraints
\item ???
\item Profit!
\end{enumerate}
\end{frame}

\subsection{Generate Constraints}

\begin{frame}{Typing relation}
\begin{block}{Relation}
\vfill
\[ \TypeRel{e}{\tauhat}{C}{R} \]
\vfill
\end{block}
\begin{block}{Legend}
\vfill
\begin{description}
\item[$\widehat\Gamma$] \quad Annotated type environment
\item[$e$]              \quad Expression being typed
\item[$\tauhat$]        \quad Type of the expression
\item[$C$]              \quad Equality constraints (e.g. $\alpha = \Bool^\varphi \to \Bool^\psi$)
\item[$R$]              \quad Subset constraints (e.g. $\{ \Nill, \Conss{\ \varphi} \} \subseteq \psi$)
\end{description}
\vfill
\end{block}
\end{frame}

\begin{frame}{Constructing Values}
\[\frac
    {\begin{gathered}\beta_1, \beta_2 \fresh\\\TypeRel{e_1}{\alpha_1}{C_1}{R_1} \quad\quad\quad\quad \TypeRel{e_2}{\alpha_2}{C_2}{R_2} \\ 
    C = C_1 \cup C_2 \cup \{\alpha_2 = \ListOf{\alpha_1}^{\beta_2}\}\\
    R = R_1 \cup R_2 \cup \{ \Conss{\beta_2} \subseteq \beta_1 \}
    \end{gathered}}
    {\begin{gathered}\TypeRel{(e_1 \textbf{\ :\ } e_2)}{\ListOf{\alpha_1}^{\beta_1}}{C}{R}\end{gathered}}
    \mbox{\ [T-Cons]}\]
\end{frame}

\begin{frame}{Pattern Matching}
\[\frac
    {\begin{gathered}\TypeRel{g}{\tauhat_\mathrm{g}}{C_\mathrm{g}}{R_\mathrm{g}}\quad\quad\quad\quad
    \beta \fresh\\
     \TypeRel{e_1}{\tauhat_1}{C_1}{R_1} \quad\quad\quad\quad
     \TypeRel{e_2}{\tauhat_2}{C_2}{R_2} \\
     C = C_\mathrm{g} \cup C_1 \cup C_2 \cup \{\tauhat_\mathrm{g} = \Bool^\beta, \tauhat_1 = \tauhat_2\} \\
     R = R_\mathrm{g} \cup R_1 \cup R_2 \cup \{\beta \subseteq \{\TTrue, \FFalse\}\}
     \end{gathered}}
    {\TypeRel{\textbf{if\ }g\textbf{\ then\ }e_1\textbf{\ else\ }e_2}{\tauhat_1}{C}{R}}
    \mbox{\ [T-If]}\]
\end{frame}

\subsection{Solve Constraints}

\begin{frame}{Overview}
\begin{itemize}
\item Solve $C$ using unification
    \begin{itemize}
    \item Includes unifying annotation variables
    \item Apply resulting substitution $\theta$ to $\tauhat$ and $R$
    \end{itemize}
\item Solve $R$ using worklist algorithm
    \begin{itemize}
    \item Do dependency analysis
    \item Determine input-independent $R' \subseteq R$
    \item Solve $R'$ using worklist algorithm
    \begin{itemize}
        \item Determines lowerbound $L$ and upperbound $U$ for all $\beta$
    \end{itemize}
    \item Check for pattern match failures ($L \subseteq U$)
    \end{itemize}
\item Generalize over $\mathsf{ftv}(\tauhat)-\mathsf{ftv}(\widehat\Gamma)$ and $R - R'$
\end{itemize}
\end{frame}

\begin{frame}{Example}
\begin{description}
\item[Intermediate result] \begin{eqnarray*} \beta_1 &=& (\{\Nill, \Conss{\Nill}, \Conss{\Conss{\Nill}}\}, \top) \\ \beta_2 &=& (\bot, \top)\end{eqnarray*}\pause
\item[Constraint\quad\quad] \[ \beta_1 \subseteq \{\Conss{\beta_2}\} \]\pause
\item[Substitute LHS] \[ \{\Nill, \Conss{\Conss{\Nill}}, \Conss{\Conss{\Conss{\Nill}}}\} \subseteq \{\Conss{\beta_2}\} \]\pause
\item[Project out fields] \begin{eqnarray*}  \{\Nill\} &\subseteq& \emptyset \\ \{\Conss{\Nill}, \Conss{\Conss{\Nill}}\} &\subseteq& \beta_2 \end{eqnarray*}
\end{description}
\end{frame}

\begin{frame}{Example}
\begin{description}
\item[Intermediate result] \begin{eqnarray*} \beta_1 &=& (\{\Nill, \Conss{\Nill}, \Conss{\Conss{\Nill}}\}, \top) \\ \beta_2 &=& (\bot, \top)\end{eqnarray*}
\item[Project out fields] \begin{eqnarray*}  \{\Nill\} &\subseteq& \emptyset \\ \{\Conss{\Nill}, \Conss{\Conss{\Nill}}\} &\subseteq& \beta_2 \end{eqnarray*}
\item[Update intermediate results] \begin{eqnarray*}L(\beta_2) &:=& L(\beta_2) \sqcup \{\Conss{\Nill}, \Conss{\Conss{\Nill}}\} \\&=& \{\Conss{\Nill}, \Conss{\Conss{\Nill}}\}\end{eqnarray*}
\end{description}
\end{frame}

\section{Conclusions}

\subsection{Conclusions}

\begin{frame}{Conclusions}
So, does it work?
\end{frame}

\begin{frame}{Further Research}
Sometimes, but there's a lot of room for improvement.
\begin{itemize}
\item Unnatural type for |map|
\item Principal types for operators
\end{itemize}
\end{frame}

\end{document}

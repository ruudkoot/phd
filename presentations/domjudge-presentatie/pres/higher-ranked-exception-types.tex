\documentclass[serif,professionalfont]{beamer}

\usepackage{etex}

%\setbeamercolor{alerted text}{fg=green,bg=red}

\usepackage[sc,osf]{mathpazo}			% Palatino (text + math)
%\usepackage{eulervm}					% Euler (math)
%\usepackage{newtxtext,newtxmath}		% Times New Roman (text + math)

\usepackage{amsmath}
\usepackage{amssymb}
\usepackage{amsthm}
\usepackage{mathtools}
\usepackage{xspace}

\usepackage[all]{xy}

\newcommand{\DOMjudge}{\textsc{DOMjudge}\xspace}

%% ODER: format ==         = "\mathrel{==}"
%% ODER: format /=         = "\neq "
%
%
\makeatletter
\@ifundefined{lhs2tex.lhs2tex.sty.read}%
  {\@namedef{lhs2tex.lhs2tex.sty.read}{}%
   \newcommand\SkipToFmtEnd{}%
   \newcommand\EndFmtInput{}%
   \long\def\SkipToFmtEnd#1\EndFmtInput{}%
  }\SkipToFmtEnd

\newcommand\ReadOnlyOnce[1]{\@ifundefined{#1}{\@namedef{#1}{}}\SkipToFmtEnd}
\usepackage{amstext}
\usepackage{amssymb}
\usepackage{stmaryrd}
\DeclareFontFamily{OT1}{cmtex}{}
\DeclareFontShape{OT1}{cmtex}{m}{n}
  {<5><6><7><8>cmtex8
   <9>cmtex9
   <10><10.95><12><14.4><17.28><20.74><24.88>cmtex10}{}
\DeclareFontShape{OT1}{cmtex}{m}{it}
  {<-> ssub * cmtt/m/it}{}
\newcommand{\texfamily}{\fontfamily{cmtex}\selectfont}
\DeclareFontShape{OT1}{cmtt}{bx}{n}
  {<5><6><7><8>cmtt8
   <9>cmbtt9
   <10><10.95><12><14.4><17.28><20.74><24.88>cmbtt10}{}
\DeclareFontShape{OT1}{cmtex}{bx}{n}
  {<-> ssub * cmtt/bx/n}{}
\newcommand{\tex}[1]{\text{\texfamily#1}}	% NEU

\newcommand{\Sp}{\hskip.33334em\relax}


\newcommand{\Conid}[1]{\mathit{#1}}
\newcommand{\Varid}[1]{\mathit{#1}}
\newcommand{\anonymous}{\kern0.06em \vbox{\hrule\@width.5em}}
\newcommand{\plus}{\mathbin{+\!\!\!+}}
\newcommand{\bind}{\mathbin{>\!\!\!>\mkern-6.7mu=}}
\newcommand{\rbind}{\mathbin{=\mkern-6.7mu<\!\!\!<}}% suggested by Neil Mitchell
\newcommand{\sequ}{\mathbin{>\!\!\!>}}
\renewcommand{\leq}{\leqslant}
\renewcommand{\geq}{\geqslant}
\usepackage{polytable}

%mathindent has to be defined
\@ifundefined{mathindent}%
  {\newdimen\mathindent\mathindent\leftmargini}%
  {}%

\def\resethooks{%
  \global\let\SaveRestoreHook\empty
  \global\let\ColumnHook\empty}
\newcommand*{\savecolumns}[1][default]%
  {\g@addto@macro\SaveRestoreHook{\savecolumns[#1]}}
\newcommand*{\restorecolumns}[1][default]%
  {\g@addto@macro\SaveRestoreHook{\restorecolumns[#1]}}
\newcommand*{\aligncolumn}[2]%
  {\g@addto@macro\ColumnHook{\column{#1}{#2}}}

\resethooks

\newcommand{\onelinecommentchars}{\quad-{}- }
\newcommand{\commentbeginchars}{\enskip\{-}
\newcommand{\commentendchars}{-\}\enskip}

\newcommand{\visiblecomments}{%
  \let\onelinecomment=\onelinecommentchars
  \let\commentbegin=\commentbeginchars
  \let\commentend=\commentendchars}

\newcommand{\invisiblecomments}{%
  \let\onelinecomment=\empty
  \let\commentbegin=\empty
  \let\commentend=\empty}

\visiblecomments

\newlength{\blanklineskip}
\setlength{\blanklineskip}{0.66084ex}

\newcommand{\hsindent}[1]{\quad}% default is fixed indentation
\let\hspre\empty
\let\hspost\empty
\newcommand{\NB}{\textbf{NB}}
\newcommand{\Todo}[1]{$\langle$\textbf{To do:}~#1$\rangle$}

\EndFmtInput
\makeatother
%
%
%
%
%
%
% This package provides two environments suitable to take the place
% of hscode, called "plainhscode" and "arrayhscode". 
%
% The plain environment surrounds each code block by vertical space,
% and it uses \abovedisplayskip and \belowdisplayskip to get spacing
% similar to formulas. Note that if these dimensions are changed,
% the spacing around displayed math formulas changes as well.
% All code is indented using \leftskip.
%
% Changed 19.08.2004 to reflect changes in colorcode. Should work with
% CodeGroup.sty.
%
\ReadOnlyOnce{polycode.fmt}%
\makeatletter

\newcommand{\hsnewpar}[1]%
  {{\parskip=0pt\parindent=0pt\par\vskip #1\noindent}}

% can be used, for instance, to redefine the code size, by setting the
% command to \small or something alike
\newcommand{\hscodestyle}{}

% The command \sethscode can be used to switch the code formatting
% behaviour by mapping the hscode environment in the subst directive
% to a new LaTeX environment.

\newcommand{\sethscode}[1]%
  {\expandafter\let\expandafter\hscode\csname #1\endcsname
   \expandafter\let\expandafter\endhscode\csname end#1\endcsname}

% "compatibility" mode restores the non-polycode.fmt layout.

\newenvironment{compathscode}%
  {\par\noindent
   \advance\leftskip\mathindent
   \hscodestyle
   \let\\=\@normalcr
   \let\hspre\(\let\hspost\)%
   \pboxed}%
  {\endpboxed\)%
   \par\noindent
   \ignorespacesafterend}

\newcommand{\compaths}{\sethscode{compathscode}}

% "plain" mode is the proposed default.
% It should now work with \centering.
% This required some changes. The old version
% is still available for reference as oldplainhscode.

\newenvironment{plainhscode}%
  {\hsnewpar\abovedisplayskip
   \advance\leftskip\mathindent
   \hscodestyle
   \let\hspre\(\let\hspost\)%
   \pboxed}%
  {\endpboxed%
   \hsnewpar\belowdisplayskip
   \ignorespacesafterend}

\newenvironment{oldplainhscode}%
  {\hsnewpar\abovedisplayskip
   \advance\leftskip\mathindent
   \hscodestyle
   \let\\=\@normalcr
   \(\pboxed}%
  {\endpboxed\)%
   \hsnewpar\belowdisplayskip
   \ignorespacesafterend}

% Here, we make plainhscode the default environment.

\newcommand{\plainhs}{\sethscode{plainhscode}}
\newcommand{\oldplainhs}{\sethscode{oldplainhscode}}
\plainhs

% The arrayhscode is like plain, but makes use of polytable's
% parray environment which disallows page breaks in code blocks.

\newenvironment{arrayhscode}%
  {\hsnewpar\abovedisplayskip
   \advance\leftskip\mathindent
   \hscodestyle
   \let\\=\@normalcr
   \(\parray}%
  {\endparray\)%
   \hsnewpar\belowdisplayskip
   \ignorespacesafterend}

\newcommand{\arrayhs}{\sethscode{arrayhscode}}

% The mathhscode environment also makes use of polytable's parray 
% environment. It is supposed to be used only inside math mode 
% (I used it to typeset the type rules in my thesis).

\newenvironment{mathhscode}%
  {\parray}{\endparray}

\newcommand{\mathhs}{\sethscode{mathhscode}}

% texths is similar to mathhs, but works in text mode.

\newenvironment{texthscode}%
  {\(\parray}{\endparray\)}

\newcommand{\texths}{\sethscode{texthscode}}

% The framed environment places code in a framed box.

\def\codeframewidth{\arrayrulewidth}
\RequirePackage{calc}

\newenvironment{framedhscode}%
  {\parskip=\abovedisplayskip\par\noindent
   \hscodestyle
   \arrayrulewidth=\codeframewidth
   \tabular{@{}|p{\linewidth-2\arraycolsep-2\arrayrulewidth-2pt}|@{}}%
   \hline\framedhslinecorrect\\{-1.5ex}%
   \let\endoflinesave=\\
   \let\\=\@normalcr
   \(\pboxed}%
  {\endpboxed\)%
   \framedhslinecorrect\endoflinesave{.5ex}\hline
   \endtabular
   \parskip=\belowdisplayskip\par\noindent
   \ignorespacesafterend}

\newcommand{\framedhslinecorrect}[2]%
  {#1[#2]}

\newcommand{\framedhs}{\sethscode{framedhscode}}

% The inlinehscode environment is an experimental environment
% that can be used to typeset displayed code inline.

\newenvironment{inlinehscode}%
  {\(\def\column##1##2{}%
   \let\>\undefined\let\<\undefined\let\\\undefined
   \newcommand\>[1][]{}\newcommand\<[1][]{}\newcommand\\[1][]{}%
   \def\fromto##1##2##3{##3}%
   \def\nextline{}}{\) }%

\newcommand{\inlinehs}{\sethscode{inlinehscode}}

% The joincode environment is a separate environment that
% can be used to surround and thereby connect multiple code
% blocks.

\newenvironment{joincode}%
  {\let\orighscode=\hscode
   \let\origendhscode=\endhscode
   \def\endhscode{\def\hscode{\endgroup\def\@currenvir{hscode}\\}\begingroup}
   %\let\SaveRestoreHook=\empty
   %\let\ColumnHook=\empty
   %\let\resethooks=\empty
   \orighscode\def\hscode{\endgroup\def\@currenvir{hscode}}}%
  {\origendhscode
   \global\let\hscode=\orighscode
   \global\let\endhscode=\origendhscode}%

\makeatother
\EndFmtInput
%
%
%
% First, let's redefine the forall, and the dot.
%
%
% This is made in such a way that after a forall, the next
% dot will be printed as a period, otherwise the formatting
% of `comp_` is used. By redefining `comp_`, as suitable
% composition operator can be chosen. Similarly, period_
% is used for the period.
%
\ReadOnlyOnce{forall.fmt}%
\makeatletter

% The HaskellResetHook is a list to which things can
% be added that reset the Haskell state to the beginning.
% This is to recover from states where the hacked intelligence
% is not sufficient.

\let\HaskellResetHook\empty
\newcommand*{\AtHaskellReset}[1]{%
  \g@addto@macro\HaskellResetHook{#1}}
\newcommand*{\HaskellReset}{\HaskellResetHook}

\global\let\hsforallread\empty

\newcommand\hsforall{\global\let\hsdot=\hsperiodonce}
\newcommand*\hsperiodonce[2]{#2\global\let\hsdot=\hscompose}
\newcommand*\hscompose[2]{#1}

\AtHaskellReset{\global\let\hsdot=\hscompose}

% In the beginning, we should reset Haskell once.
\HaskellReset

\makeatother
\EndFmtInput

\setbeamersize{description width=-\labelsep}

\begin{document}

\title{Datastructuren 2014--2015}
\subtitle{Introductie practicum en \DOMjudge}
%\author{Ruud Koot}
%\institute[UU]{Department of Information and Computing Sciences\\Utrecht University}
\date{22 april 2015}
\maketitle

\begin{frame}{Practicum}

    \begin{itemize}
    
        \item Er zijn 7 opgaven.
        
        \item Opgave 1 telt voor 0.4 punt; elk van de andere opgaven telt voor 1.6 punt.
        \item Een goede opgave levert een 10 op; een foute een 0.
        
        \item Inleveren gaat met behulp van \DOMjudge.
    
    \end{itemize}
    
\end{frame}

\begin{frame}{\DOMjudge}

    \begin{itemize}
    
        \item \DOMjudge is een semi-automatisch nakijksysteem voor programmeeropgaven.
        
        \item Je levert je opgave in via een webpagina en \DOMjudge vertelt je op de je opgave fout is (waarschijnlijk) correct.
        
        \item \DOMjudge kan niet alle fouten vinden, dus er wordt ook nog door een mens naar gekeken:
        
            \begin{itemize}
            
                \item We vroegen je om \textsc{Insertion Sort} te implementeren, maar je hebt \textsc{Merge Sort} ge\"implementeerd.
                
                \item We vroegen je om \textsc{Insertion Sort} te implementeren, maar je roept het sorteer algoritme uit de standaardbibliotheek aan.
                
                \item We vroegen je om \textsc{Insertion Sort} te implementeren, maar je hebt het sorteer algoritme van internet of een andere student ingezonden.
            
            \end{itemize}
    
    \end{itemize}

\end{frame}

\begin{frame}{Inzenden}

    \begin{itemize}

        \item Te vinden op: \url{https://domjudge.cs.uu.nl/ds/team/}
        
            \begin{itemize}
            
                \item Inloggen kan met je Solis-id en wachtwoord.
                
                \item Als je nog niet ingeschreven bent mail dan naar \texttt{r.koot@uu.nl}.

            \end{itemize}
        
        \item Stuur je \emph{broncode} in en niet je al gecompileerde executable.
        
            \begin{itemize}
            
                \item \DOMjudge compileert je code zelf.
                
                \item Zo kunnen wij ook nog naar jou code kijken.
            
            \end{itemize}
        
        \item Je kan zo vaak insturen als je wilt (in het redelijke).
        
        \item De nakijkers kijken alleen naar je laatste (correcte) inzending.
        
    \end{itemize}

\end{frame}

\begin{frame}{Beoordeling}

    \begin{itemize}

        \item Enkele minuten na het insturen krijg je een antwoord:
    
        \begin{description}
        
            \item[\textsc{Pending}] Je programma staat nog in de wachtrij of wordt nu getest.
        
            \item[\textsc{Correct}] Jeej! (Mits de nakijkers het er later ook goed vinden.)
            

            \item[\textsc{Compile-Error}] Het lukte \DOMjudge niet je programma te compileren.

            \item[\textsc{Run-Error}] Je programma is gecrasht tijdens het testen.

            \item[\textsc{Time-Limit}] Je programma is te langzaam.

            \item[\textsc{Wrong-Answer}] Je programma gaf het verkeerde antwoord.
            
            \item[\textsc{No-Output}] Je programma crashte niet, maar gaf ook nooit een antwoord.
            
        \end{description}
        
    \end{itemize}

\end{frame}

\begin{frame}{Beoordeling}

    \begin{itemize}
    
        \item In het geval van een \textsc{Compile-Error}, \textsc{Run-Error}, \textsc{Time-Limit} of \textsc{Wrong-Answer} kun je op je inzending klikken voor meer details over wat er mis ging.
        
        \item Van sommige testen krijg je te zien:
        
            \begin{itemize}
            
                \item wat de vraag was,
                \item welk antwoord verwacht werd, en
                \item welk antwoord jou programma gaf.
             
            \end{itemize}
        
        \item Andere testen zijn geheim en krijg je nooit te zien.
        
        \item Als alle openbare testen goed lijken te gaan, maar je krijgt toch een \textsc{Run-Error}, \textsc{Time-Limit} of \textsc{Wrong-Answer}, dan kan dat door de geheime testen komen.
    
    \end{itemize}

\end{frame}

\begin{frame}{Assistentie}

    \begin{itemize}
    
        \item Er is een wekelijks vragenuurtje.
        
        \item Je kunt een ``clarification'' verzoek sturen via \DOMjudge.
        
            \begin{itemize}
            
                \item Stuur je code altijd in, dan kunnen wij die bekijken.
            
            \end{itemize}
        
        \item We proberen die tijdig te beantwoorden, maar niet op de dag van de deadline.
    
    \end{itemize}

\end{frame}

\begin{frame}{Tips \& truuks}

    \begin{itemize}
    
        \item Test je programma ook lokaal, voordat je hem instuurt.
        
            \begin{itemize}
            
                \item Er staat soms testinvoer in de opgavebeschrijving.
                
            \end{itemize}
        
        \item Verzin en maak je eigen testinvoer:
        
            \begin{itemize}
            
                \item Randgevallen
                
                \item Worst-case gevallen
            
            \end{itemize}
        
        \item We zijn ge\"intereseerd in de asymptotische looptijd van je programma, maar kunnen dit niet nauwkeurig meten:
        
            \begin{itemize}
            
                \item Doe geen onnodig langzame dingen, zoals een enorm aantal objecten aanmaken, waar een primitief type of array volstaat.
            
            \end{itemize}

        \item \DOMjudge is geen gokkast: maak geen willekeurige aanpassingen, maar beredeneer waarom je programma wel/niet werkt.

        \item Gebruik een versiebeheersysteem, bijvoorbeeld \texttt{git}.
    
    \end{itemize}

\end{frame}

\begin{frame}{Technische details}

    \begin{description}
    
        \item[C$\sharp$] Mono 3.2.8
        
        \item[Java] Oracle Java 7
    
    \end{description}

\end{frame}

\end{document}
